\chapter{Objetivos}

\section{Objetivos acad�micos}
\noindent

Una vez finalizados los estudios del Grado en Ingenier�a en Inform�tica en la Escuela de Ingenier�a Inform�tica perteneciente a la Universidad de Las Palmas de Gran Canaria, y con objeto de obtener la titulaci�n que acredite los mencionados estudios se ha desarrollado, con el apoyo del tutor, el presente proyecto cuyo t�tulo es:

\begin{center}
\textbf{...} 
\end{center} 


\section{Objetivos generales del proyecto}
\noindent

Los objetivos que se persiguen en este proyecto se resumen en dos. En primer lugar, abordar el desarrollo e implementaci�n de los algoritmos necesarios para llevar a cabo la inserci�n de objetos geom�tricos en una secuencia v�deo. En segundo lugar, desarrollar e implementar un interfaz de usuario que permita su utilizaci�n por un potencial usuario final.

La implementaci�n del proyecto tiene dos factores cr�ticos: el tiempo de c�mputo y el uso de recursos. 

Los objetivos generales se deben descomponer y asignar unos objetivos parciales a cada etapa para que el cumplimiento de �stos, repercuta en la consecuci�n del objetivo del nivel superior.

Los principales objetivos que se han perseguido en cada etapa del proyecto son la precisi�n y fiabilidad. Los resultados de cada etapa ser�n los datos de entrada en la siguiente. Los errores, en el caso de producirse, se propagan y repercuten en los resultados finales.
\\

...

