% Definimos el estilo del documento
\documentclass[12pt,a4paper,spanish]{book}

% Utilizamos el paquete para utilizar espa�ol
\usepackage[spanish]{babel}
%\selectlanguage{spanish}
%\hyphenation{re-cons-trui-do}

% Utilizamos el paquete para gestionar acentos
%\usepackage[T1]{fontenc}
\usepackage[latin1]{inputenc}

%Utilizamos el paquete para incorporar gr�ficos postcript
\usepackage[dvips,final]{epsfig}
%\usepackage{psfig}
\usepackage{color}
\usepackage{graphicx}
\usepackage{amsmath}%
\usepackage{amssymb}%

% Portada
%\usepackage{xthesis}  % Chabi Thesis Style

% Definimos la zona de la pagina ocupada por el texto
%\oddsidemargin 0.1cm \headsep 0.5cm \textwidth=15.5cm
%\textheight=22cm

% Definimos el formato de la bibliograf�a (referencias)
\bibliographystyle{alpha}

\sloppy

%Empieza el documento
\begin{document}

% ************************************************************
% Definimos titulo, autor, fecha, generamos titulo e �ndice de contenidos
\title{\textsf{Titulo del proyecto} \\ \rule{13cm}{0.1cm} }
\author{Autor del proyecto}
\date{}
\maketitle

\tableofcontents
\listoffigures
\listoftables

% Definimos una primera pagina para los agradecimientos
\newpage
\thispagestyle{empty}
\section*{Agradecimientos}
%  Aqu� ponemos los agradecimientos

Deseo dedicar este proyecto ...
\\

Y, en general, agradecer a todas las personas que han intervenido directa o indirectamente en el proyecto.

% Empezamos cap�tulos
\chapter{Introducci\'on}

...

El gran avance en los �ltimos a�os de la tecnolog�a digital de im�genes, la aparici�n del v�deo digital, el enorme 
aumento en las prestaciones de los ordenadores y el desarrollo de la inform�tica gr�fica que permite la generaci�n de objetos y escenarios sint�ticos, ha permitido el desarrollo de una nueva tecnolog�a que permite la generaci�n de v�deos digitales combinando im�genes 
reales y sint�ticas. Las aplicaciones de este tipo de tecnolog�a son muy variadas, podemos destacar los efectos especiales 
en cine, publicidad, la generaci�n de estudios sint�ticos para televisi�n, simulaci�n del impacto visual de un edificio en un 
entorno real, etc.. Es en este contexto donde se enmarca el proyecto fin de carrera presentado en esta memoria.  
  

...  
  

% Estructura de los cap�tulos
Siguiendo esta l�nea general de trabajo que pretendemos abordar, hemos estructurado el contenido de este documento de la siguiente forma:  

Comenzaremos analizando el entorno del proyecto. En los primeros cuatro cap�tulos hablaremos, en primer lugar del \textit{estado actual del tema} haciendo referencia a trabajos anteriores en los que nos hemos basados en nuestro proyecto y herramientas relacionadas con algunas etapas del proyecto. En el cap�tulo de \textit{objetivos} se hace una descripci�n general de los retos que se pretender alcanzar en cada una de las etapas en las que est� dividido el proyecto. En el cap�tulo de \textit{recursos} se comenta los principales recursos \textit{hardware} y \textit{software} empleados. En el software se har� una distinci�n entre las aplicaciones finales que hemos utilizado, y las librer�as que nos han ayudado a desarrollar los componentes software de nuestro proyecto. En la \textit{planificaci�n} se hablar� sobre la estimaci�n inicial prevista y el tiempo real invertido. Se comentar�n algunos aspectos que han influido notablemente en esta diferencia.
\\


A continuaci�n, se abordar� en detalle la descripci�n de cada una de las etapas en las que se ha divido el proyecto. Esta descripci�n se ha dividido en cinco cap�tulos, siguiendo un desglose por etapas, y supone el eje central de la memoria. 

En la \textit{primera etapa} se har� una breve introducci�n a los distintos tipos de detectores de puntos singulares que existen, destacando sus principales caracter�sticas. Describiremos la teor�a matem�tica que subyace al detector de Harris. Y por �ltimo, se mostrar�n y analizar�n los resultados experimentales que se han obtenido con este detector. En la \textit{segunda etapa} se estudia el funcionamiento del proceso de seguimiento de puntos singulares a trav�s de una secuencia de v�deo; estructurando el cap�tulo en tres partes: definici�n de conceptos, criterios de sucesi�n y la descripci�n del proceso de seguimiento. Al final del cap�tulo se comentar�n los resultados experimentales y diversas cuestiones a tener en cuenta. En la \textit{tercera etapa} se hace una descripci�n del problema de calibraci�n, tomando como base el modelo \textit{pin-hole} y analizando los algoritmos de calibraci�n que se utilizar�n en el proyecto. En la \textit{cuarta etapa} se analiza el proceso de inserci�n de un objeto 3D en el mundo. Este proceso se divide en tres partes: insertar el objeto, renderizar la escena y la generaci�n de la secuencia de v�deo. En este cap�tulo se har� una exhaustiva descripci�n de las alternativas planteadas en esta etapa y los diversos problemas que finalmente se solventaron. En la \textit{quinta etapa} se describe el proceso de creaci�n de la interfaz de usuario de la aplicaci�n. Este cap�tulo incluye una fase de an�lisis donde se plantean las caracter�sticas del software; otra de dise�o en la que se define la estructura de la aplicaci�n (datos e interfaz) y por �ltimo, se comenta detalles de implementaci�n de relevancia.
\\


Finalmente se exponen los resultados y conclusiones obtenidas del proyecto. Se plantear� los posibles trabajos futuros que se dejaron pendientes por cuestiones de tiempo y esfuerzo.
\\


Hemos a�adido un ap�ndice a la memoria que se compone de un �nico cap�tulo en el que se explican con todo lujo de detalles la implementaci�n de cada una de las etapas del proyecto, incluyendo las alternativas y problemas que se nos plantearon, as� como las soluciones adoptadas.


% *) ESTADO ACTUAL Y OBJETIVOS: situaci�n actual del tema relacionado con el TFT y motivaci�n de los objetivos que se pretenden cubrir con este trabajo.
\chapter{Estado actual del tema}

En este cap�tulo se describe el estado actual del tema con referencia a trabajos anteriores o relacionados con nuestro proyecto.

\chapter{Objetivos}

\section{Objetivos acad�micos}
\noindent

Una vez finalizados los estudios del Grado en Ingenier�a en Inform�tica en la Escuela de Ingenier�a Inform�tica perteneciente a la Universidad de Las Palmas de Gran Canaria, y con objeto de obtener la titulaci�n que acredite los mencionados estudios se ha desarrollado, con el apoyo del tutor, el presente proyecto cuyo t�tulo es:

\begin{center}
\textbf{...} 
\end{center} 


\section{Objetivos generales del proyecto}
\noindent

Los objetivos que se persiguen en este proyecto se resumen en dos. En primer lugar, abordar el desarrollo e implementaci�n de los algoritmos necesarios para llevar a cabo la inserci�n de objetos geom�tricos en una secuencia v�deo. En segundo lugar, desarrollar e implementar un interfaz de usuario que permita su utilizaci�n por un potencial usuario final.

La implementaci�n del proyecto tiene dos factores cr�ticos: el tiempo de c�mputo y el uso de recursos. 

Los objetivos generales se deben descomponer y asignar unos objetivos parciales a cada etapa para que el cumplimiento de �stos, repercuta en la consecuci�n del objetivo del nivel superior.

Los principales objetivos que se han perseguido en cada etapa del proyecto son la precisi�n y fiabilidad. Los resultados de cada etapa ser�n los datos de entrada en la siguiente. Los errores, en el caso de producirse, se propagan y repercuten en los resultados finales.
\\

...



% *) JUSTIFICACI�N DE LAS COMPETENCIAS ESPEC�FICAS CUBIERTAS: indicar, para las especialmente relacionadas con el trabajo desarrollado, c�mo se han cubierto con este TFT.
% *) APORTACIONES: justificar qu� es lo que este TFT aporta a nuestro entorno socio-econ�mico, t�cnico o cient�fico.
\chapter{Competencias Espec�ficas Cubiertas}

*) JUSTIFICACI�N DE LAS COMPETENCIAS ESPEC�FICAS CUBIERTAS: indicar, para las especialmente relacionadas con el trabajo desarrollado, c�mo se han cubierto con este TFT.

\chapter{Aportaciones del TFG}

*) APORTACIONES: justificar qu� es lo que este TFT aporta a nuestro entorno socio-econ�mico, t�cnico o cient�fico.


\chapter{Recursos utilizados}

Durante el desarrollo del proyecto se ha utilizado una serie de recursos para llevar a cabo los objetivos. Los recursos se pueden distinguir en cuatro grupos: \textit{hardware}, \textit{software}, \textit{librer�as} y \textit{lenguajes}.   

\section{Hardware}
\noindent

Los recursos hardware de este proyecto, son muy b�sicos. Bastaba simplemente con un ordenador personal. Se ha utilizado dos ordenadores, una estaci�n de trabajo con dos pentium III Xe�n a 1,7GHz y un ordenador personal pentium III a 1,1GHz. La estaci�n se utilizaba, principalmente, para realizar los c�lculos de la calibraci�n ya que eran los que mayor tiempo de c�mputo requer�an. En el ordenador personal se hizo el dise�o e implementaci�n de todos los apartados del proyecto, ya que no se requer�a un gran potencia de c�lculo.

Por otro lado, tambi�n se cont� con una c�mara digital con la que tomar las secuencias de im�genes.

% -----
% ------------------------------
% -----


\section{Software}
\noindent

En la descripci�n de los recursos se ha hecho una distinci�n entre el \textit{software} y \textit{librer�as}. A nuestro parecer, el software es un conjunto de herramientas que nos permiten realizar el proyecto, y nunca van a formar parte de �l. A diferencia del software, las librer�as van a quedar ligadas al proyecto de manera indefinida.

\subsubsection{Sistema Operativo}
\noindent

El sistema operativo sobre el que se apoya la aplicaci�n y las librer�as del proyecto es Linux. \textit{Linux} es un poderoso y sumamente vers�til Sistema Operativo de 32 bits, multi-usuario y multi-�rea. Fue creado en 1991 por Linus Torvalds, siendo entonces un estudiante de la Universidad de Helsinki. Linus se bas� sobre \textit{Unix}.

Linux y toda la comunidad que gira alrededor de �l hacen que sea una buena plataforma de desarrollo. Por un lado, se dispone un sistema operativo estable, seguro y en constante evoluci�n. Por otro lado, existe las herramientas gratuitas de todo tipo y que adem�s son de alta calidad.



\subsubsection{...}
\noindent



% -----
% ------------------------------
% -----

\section{Librer�as}
\noindent

Las librer�as son un software que resuelven un problema al programador, ocult�ndole la complejidad subyacente. De esta manera, el programador se apoya en las librer�as para resolver su problema (de m�s alto nivel). Veamos las principales librer�as que se han utilizado a lo largo del proyecto. La mayor�a de ellas, en las dos �ltimas etapas.

\subsubsection{STL}
\noindent

STL es la librer�a de plantillas est�ndar \textit{(Standard Template Library)}. Sus principales componentes son \textit{contenedores}, \textit{algoritmos}, \textit{iteradores} y \textit{objetos funci�n}. Esta librer�a es parte del est�ndar ISO C++. STL es m�s que una librer�a, es un marco de trabajo (\textit{framework}) y un paradigma en la programaci�n.

Todas las estructuras contenedores del proyecto se han implementado usando \textit{STL}. Como ya se describir� en el ap�ndice de la memoria las ventajas que aporta son m�ltiples.


\subsubsection{...}
\noindent


% -----

\section{Otras herramientas utilizadas}
\noindent

\subsubsection{Lenguaje de programaci�n C/C++}
\noindent

El lenguaje de programaci�n por el que se ha optado para la implementaci�n de este proyecto ha sido C++, ya que se pens� que un lenguaje orientado a objetos era la forma m�s adecuada de abordar las metas establecidas.

Este lenguaje es un superconjunto del lenguaje conocido como C, uno de los m�s extendidos en el mundo inform�tico, utilizado por ejemplo para la implementaci�n de sistemas operativos como Linux o la librer�a Qt. Fue dise�ado por Bjarne Stroustrup a principios de los ochenta en AT\&T Bell Laboratories, incrementando las caracter�sticas de C y a�adiendo recursos para la programaci�n orientada a objetos.

El paradigma de la programaci�n orientada a objetos es ofrecer una serie de facilidades caracterizadas por:

\begin{itemize}
\item La abstracci�n de datos, habilidad de construir estructuras de datos para definir un objeto y usarlos dentro de un programa sin tener que atender a sus detalles internos.

\item El encapsulamiento, que suministra el reforzamiento necesario para que s�lo las funciones asociadas a las estructuras tengan acceso a los detalles internos.

\item La herencia, que permite al programador definir nuevos objetos en t�rminos de objetos previamente definidos. La habilidad de ser tambi�n objeto de la clase base se denomina \textit{polimorfismo}. 
\end{itemize} 

C++ es cien por cien compatible con C. Es decir, se puede usar cualquier c�digo escrito en C en programas desarrollados en C++ sin ning�n problema. Por lo tanto, ofrece todas las posibilidades de C y algunos a�adidos como la abstracci�n de datos, verificaci�n estricta de tipos, paso de argumentos por referencia, sobrecarga de operadores y funciones, declaraci�n de las variables en cualquier lugar, tipos gen�ricos o parametrizados (templates), tratamiento de excepciones y conversi�n de tipos.

Una vez elegido el lenguaje de programaci�n deb�a elegir el compilador a usar. El GNU C++ es el compilador de C++ que viene por defecto en todas las distribuciones de Linux. 



\subsubsection{UML}
\noindent

El Lenguaje Unificado de Modelado (UML, Unified Modeling Language) es un lenguaje gr�fico para visualizar, especificar, construir y documentar los artefactos de un sistema con gran cantidad de software. UML proporciona una forma est�ndar de escribir los planos de un sistema, cubriendo tanto las cosas conceptuales, tales como procesos del negocio y funciones del sistema, como las cosas concretas, tales como las clases escritas en un lenguaje de programaci�n espec�fico, esquemas de bases de datos y componentes software reutilizables.

UML se ha utilizado en el proyecto para modelar la aplicaci�n principal del proyecto. Apoy�ndonos en el UML se extrajo una primera aproximaci�n de los requerimientos y de c�mo deb�a ser la estructura de la aplicaci�n.


\subsubsection{\LaTeX{}}
\noindent

\LaTeX{} es un paquete de macros que le permite al autor de un texto componer e imprimir sus documentos de un modo sencillo, con la mayor calidad tipogr�fica, utilizando para ello patrones previamente definidos. \LaTeX{} desempe�a el papel del dise�ador tomando parte en el formato del documento (longitud del rengl�n, tipo de letra, espacios, etc) para darle luego instrucciones a \TeX.

El tratamiento del texto es totalmente diferente a procesadores tales como \textit{Microsoft Word}, \textit{Word Perfect} o \textit{Framemaker} en los cuales el autor ve en la pantalla exactamente lo que luego aparecer� por la impresora. Esto tiene sus ventajas e inconvenientes. A�n as� \LaTeX{} tiene la posibilidad, una vez procesado el fichero, de ver el resultado final por la pantalla.   

\LaTeX{} se ha utilizado para crear toda la documentaci�n del proyecto.



% -----

%********************************************************+********************************************************
%********************************************************+********************************************************

\section{Requerimientos del software}
\noindent

Una vez le�do, cu�les han sido los recursos m�s significativos que se han utilizado para llevar a cabo todas las etapas del proyecto, veamos los requerimientos del producto final, AMICam.

\begin{itemize}
\item Sistema operativo Linux.

\item Librer�a Qt, versi�n 3.0 � superior.

\item Open Inventor, versi�n 2.1.5 en la implementaci�n de SGI.

\item SoQt, versi�n 1.0 � superior.

\end{itemize} 

\chapter{Plan de trabajo y temporizaci�n}

En este cap�tulo se presenta el plan de trabajo desglosado en etapas, con una estimaci�n en cada etapa del tiempo de ejecuci�n.

\begin{itemize}
\item Etapa 1. ... 

El tiempo invertido en esta etapa fue de 40 horas. 


\item Etapa 2. 

El tiempo invertido en esta etapa fue de 380 horas. 


\item Etapa ...

El tiempo empleado en esta etapa fue de 100 horas. 

\end{itemize}


El tiempo invertido en la escritura de esta documentaci�n fue de 200 horas.
\\

Por tanto, el tiempo total de dedicaci�n al proyecto ha sido de unas xxxx horas.


% *) DESARROLLO (REQUISITOS, DISENO, ...)
\chapter{Etapa1. Algoritmo de Harris}

.........

La primera etapa del proyecto consiste en la implementaci�n de un algoritmo para la extracci�n de puntos singulares en una imagen. En especial nos centraremos en el algoritmo de Harris. Este detector de puntos destaca por su rapidez de c�mputo y la precisi�n de los datos obtenidos.

En este cap�tulo haremos una breve introducci�n a los distintos tipos de detectores de puntos singulares que existen, destacando sus principales caracter�sticas. Describiremos la teor�a matem�tica que subyace al detector de Harris. Por �ltimo, se mostrar�n y analizar�n los resultados experimentales que se han obtenido.


\section{Detectores de esquinas}
\noindent

Una imagen contiene una gran cantidad de datos, la mayor�a de los cuales proporciona muy poca informaci�n para interpretar la escena. Nuestro proceso debe extraer de la forma m�s eficaz y robusta posible determinadas caracter�sticas que nos proporcionen la m�xima informaci�n. Estas caracter�sticas deben cumplir, entre otras, las siguientes condiciones:

\begin{itemize}
  \item Su extracci�n a partir de la imagen no debe suponer un coste excesivo. El tiempo total de extracci�n debe ser lo m�s   peque�o posible.

  \item Su localizaci�n debe ser muy precisa. El error cometido en la estimaci�n de las caracter�sticas tambi�n debe ser lo   m�s peque�o posible.

  \item Deben ser robustas y estables. Deber�an permanecer a lo largo de una secuencia.

\end{itemize}


Las esquinas de los objetos presentes en la imagen satisfacen estas condiciones. Aparecen de forma natural en la mayor�a de escenarios tanto naturales como artificiales. Un punto esquina es un punto del contorno de un objeto donde la curvatura es alta.

Desde el punto de vista computacional, se han propuesto una serie de enfoques para la detecci�n de este tipo de caracter�sticas bidimensionales. Podemos clasificar estos enfoques en dos grupos principales:

\begin{enumerate}
  \item M�todos que obtienen las aristas de la imagen mediante alg�n m�todo de detecci�n de aristas, para, a continuaci�n,   detectar puntos de cruce entre aristas o con un cambio sustancial en la direcci�n de la arista. Estos puntos se clasifican como puntos esquina.

  \item El resto de m�todos trabajan directamente sobre las im�genes, es decir, las esquinas no se infieren a partir de la extracci�n de aristas.

\end{enumerate}


\paragraph{Puntos de inter�s extra�dos a partir de aristas}
\noindent

Todos los m�todos de obtenci�n de esquinas y/o uniones dentro de este grupo tienen en com�n una primera fase de extracci�n de aristas. Algunos m�todos de extracci�n de aristas pueden ser {\it Torre-Poggio\/}, {\it Canny\/} o {\it Smith-Brady\/}. En general, la eficiencia de estos m�todos dependen directamente de la calidad del m�todo empleado para la obtenci�n de aristas: si este �ltimo no localiza correctamente los puntos de arista, dif�cilmente podremos detectar puntos de esquina de forma exacta. Adem�s se a�ade un tiempo extra de procesamiento que en ciertos sistemas puede ser prohibitivo.

\chapter[Seguimiento de puntos]{Seguimiento de puntos singulares a trav�s de una secuencia de v�deo}

En este cap�tulo vamos a estudiar el funcionamiento del proceso de seguimiento de puntos singulares a
trav�s de una secuencia de v�deo. El seguimiento consiste en hallar la trayectoria de los puntos singulares 
a trav�s de los distintos frames. La trayectoria de un punto 3d consiste en una lista con las proyecciones 
de dicho punto en los distintos frames donde es visible.

Este cap�tulo se divide en tres partes: definici�n de conceptos, criterios de sucesi�n y la descripci�n del
proceso de seguimiento. El apartado de definici�n de conceptos es imprescindible para introducir al lector
en la terminolog�a que se maneja. Los criterios de sucesi�n definir�n el comportamiento que tendr� el proceso
de seguimiento.



\section{Proceso del seguimiento}
\noindent

\subsection{Conceptos}
\noindent

A lo largo de los siguientes apartados manejaremos una serie de conceptos por lo que necesitaremos
entenderlos perfectamente. A continuaci�n se definen dichos conceptos:


\begin{itemize}
\item Detector de esquinas:
\noindent
Herramienta software que nos identifica un conjunto de esquinas (extremos de curvatura) en una imagen.

\item Frame:
\noindent
Instant�nea que recoge informaci�n captada por una c�mara (ya sea de v�deo o de fotograf�a).
%Instant�nea extraida de un v�deo. En nuestro caso, ser� una determinada imagen de la secuencia de entrada.

\item Par�metros de Harris:
\noindent
El algoritmo de Harris es el detector de esquinas que utilizaremos el seguimiento de los puntos.
Sus par�metros han sido descritos en el cap�tulo anterior.

\item Punto sucesor:
\noindent
Punto continuaci�n de otro situado en el frame anterior que corresponde al mismo punto 3d.

\item Secuencia de puntos:
\noindent
Valor devuelto en el seguimiento, en que se identifica como una lista de puntos aquellas proyecciones de
frames consecutivos que correspondan al mismo punto 3d.

\item Secuencia de v�deo:
\noindent
Conjunto de frames desglosados uno independiente del otro.

\end{itemize}



\include{memoria_etapa3}
\include{memoria_etapa4}
\include{memoria_etapa5}

% *) ASPECTOS ECON�MICOS Y TEMPORALES
\chapter{Resultados}

En cada uno de los cap�tulos anteriores se han comentado de manera exhaustiva los resultados emp�ricos obtenidos en cada etapa. En este �ltimo apartado vamos a globalizar el resultado final aplicado a una secuencia de v�deo.

% Descripci�n de las condiciones en la que se tom� la secuencia
Partimos de una secuencia de 120 im�genes reales de un despacho. Cada una de estas im�genes fueron tomadas por una c�mara digital con un zoom constante. La c�mara est� situada en el fondo del despacho y se va a ir desplazando paralelo a la pared. El movimiento seguido por la c�mara se asemeja a una recta. La diferencia entre dos im�genes consecutivas es peque�a. Al no tomarse esta secuencia con una c�mara de v�deo digital, se observar�n peque�o saltos entre im�genes por lo que el movimiento no ser� fluido.

La c�mara se ha situado encima de un soporte m�vil y es �ste quien se mueve. De esta manera se evita el cabeceo de la c�mara. La fuente de luz en la escena est� situada detr�s de la c�mara y la imagen tomadas son de buena calidad y de gran claridad. 

En las figuras \ref{fig_despacho_inicial_0_30}, \ref{fig_despacho_inicial_50_90} y \ref{fig_despacho_inicial_100_119} podemos observar una mini secuencia de seis im�genes. Estas im�genes son un subconjunto de secuencia de v�deo inicial (120). Cada una de estas im�genes pertenece a distintas partes de esta secuencia. En concreto se han tomado las im�genes de las posiciones 0, 30 50, 90, 100 y 119.


% % Secuencia Inicial
% \begin{figure}
%    \begin{center}
%      \includegraphics[width=14cm ]
% 	  {imagen_ini_1_30.ps}
%      \caption{De arriba a abajo tenemos las im�genes de las posiciones 0 y 30 de la secuencia, respectivamente.}
%      \label{fig_despacho_inicial_0_30}
%    \end{center}
% \end{figure}
% 



% *) CONCLUSIONES Y TRABAJOS FUTUROS
\chapter{Conclusiones}
\noindent

El proyecto fin de carrera tiene por objetivo la elaboraci�n de un proyecto inform�tico aplicando los conocimientos adquiridos a lo largo de las asignaturas de la Ingenier�a en Inform�tica.

La inserci�n de objetos geom�tricos en una secuencia de v�deo, persegu�a dos objetivos muy claros: (1) desarrollo e implementaci�n de los algoritmos necesarios para llevar a cabo esta tarea y (2) desarrollar e implementar un interfaz de usuario que permitiera su utilizaci�n por un potencial usuario final.

Como se ha visto a lo largo de esta memoria, los objetivos iniciales han sido superados, ya que, no s�lo se han implementado los requisitos de cada etapa del proyecto, sino que:

\begin{itemize}
\item Cada etapa del proyecto se ha dise�ado utilizando una estructura modular, facilit�ndose su mantenimiento y reutilizaci�n.

\item La funcionalidad prevista se ha aumentado con el fin de que el usuario disponga del mayor n�mero de herramientas posible.

\item Se han optimizado ciertos procesos del proyecto con el objetivo de reducir el tiempo de c�mputo y el uso de recursos.
\end{itemize} 


% Tiempo transcurrido
El tiempo transcurrido desde la elecci�n del proyecto hasta la finalizaci�n del mismo ha sido de 18 meses. De ese tiempo, alrededor de 11 meses es el tiempo total invertido en el proyecto. El resto del tiempo se ha empleado en el aprendizaje de numerosas herramientas como son las librer�as Qt y Open Inventor. 

Con respecto al dise�o, hemos tenido que replantear en varias ocasiones las estrategias utilizadas lo que ha supuesto un 
aumento considerable de horas de trabajo respecto al plan inicial previsto. Tambi�n nos  
hemos tenido que enfrentar a numerosos problemas que a priori no sab�amos si ten�an soluci�n. 
Las etapas en las que m�s tiempo se invirti� fueron la de la creaci�n del interfaz y la de inserci�n de un 
objeto en la escena. La creaci�n del interfaz es un proceso largo y laborioso. El desarrollo del software se 
plante� siguiendo cada una de las fases que dicta la ingenier�a del software. 
\\

Un punto a destacar es que la creaci�n de una interfaz amigable y de f�cil manejo como es AMICam, implica la gesti�n de n�merosos eventos y detalles. Esa funcionalidad se traduce en tiempo de desarrollo y depuraci�n. Finalmente, AMICam se convirti� en una interfaz donde el peso de la gesti�n recae en el sistema y no en el usuario. En el proyecto nos hemos centrado en el sistema operativo Linux. La posibilidad de ser ejecutada sobre el sistema operativo Windows no se tuvo en cuenta ya que no formaba parte de los objetivos del proyecto. A priori el proyecto es totalmente transportable a otras plataformas. Sin embargo no estoy seguro si la librer�a de calibraci�n utiliza otras librer�as que s�lo est�n implementadas para Linux.
\\

Sin embargo, la etapa m�s compleja debido al desconocimiento de la herramienta y c�mo resolver el problema que se planteaba, fue sin dudas la inserci�n de un objeto geom�trico en la escena. El hecho de encajar la proyecci�n de una c�mara real con la de una c�mara artificial, fue el problema m�s importante que se abord� en el proyecto. Tal fue la complejidad, que se plante� la posibilidad de crear un peque�o motor de render. Debido a las expectativas que ofrec�a Inventor, se insisti� y al final se consigui� este objetivo.
\\


....


%Final
En definitiva, el desarrollo de este proyecto ha significado un gran objetivo a nivel personal, he aprendido y comprendido todos los conceptos y problemas que rodean la combinaci�n de im�genes reales y sint�ticas en una secuencia v�deo.

\chapter{Trabajo Futuro}

Durante este proyecto fin de carrera se nos han planteado otros trabajos que se pod�an desarrollar con el fin de mejorar los resultados parciales y finales, pero que por cuestiones de tiempo y esfuerzo no hemos podido realizar. Entre estos podemos destacar:
\\

\textbf{Ampliar los formatos del v�deo}.
En el proyecto se considera el v�deo como una secuencia de im�genes separadas y en formato imagen. Ser�a deseable que AMICam soportara un mayor n�mero de formatos de v�deo (tanto nuevos formatos de im�genes como formatos de v�deo digital). En especial, el trabajo futuro deber�a estar orientado a dar soporte de v�deo digital. El �nico cambio a realizar en la aplicaci�n ser�a el de incorporar librer�as que leyesen el v�deo. Se debe dar soporte, al menos, a los formatos m�s conocidos como son el \textit{.avi} y el \textit{.mpeg}. 

Se espera que la pr�xima versi�n de la librer�a gr�fica \textit{Inventor}, de soporte para grabar secuencias de v�deo en formato \textit{mpeg}.
\\


% EMPIEZAN LOS AP�NDICES DEL PROYECTO
\appendix

% *) MANUAL DE USUARIO Y SOFTWARE
%\include{memoria_manuales}
\chapter[Detalles de implementaci�n]{Detalles t�cnicos sobre la implementaci�n del proyecto}

En este cap�tulo se expone todo lo relacionado con los diferentes detalles t�cnicos de la implementaci�n del proyecto. Se
comienza introduciendo las decisiones m�s importantes que se han tomado a lo largo de las fases de dise�o, desarrollo y
pruebas.

Dada la naturaleza gr�fica del proyecto, se escogi� para la implementaci�n el sistema operativo Linux y su entorno {\it X
Window \/}, frente por ejemplo al entorno MS-Windows, tanto por la disponibilidad de m�quinas sobre las que trabajar, de las
cuales he obtenido gran experiencia, como de las ventajas en cuanto a fiabilidad y posibilidad de portar el c�digo a otras
implementaciones de S.O. que a su vez ofrecen, en su conjunto, un amplio abanico de plataformas hardware.

El XMegaWave es un entorno, con una interfaz gr�fica, para el procesamiento gr�fico de im�genes. El usuario puede interactuar
con la interfaz mediante rat�n y pop-up men�s. Adem�s se puede incluir en el entorno funciones creadas por propio usuario.
El XMegaWave puede trabajar con v�deos (en un formato especial ) y con im�genes, tanto en color como en escala de grises.
Est� orientado a estaciones UNIX con X11 y librer�as gr�ficas Motif.




\section{Uso de las STL (Standard Template Library)}
\noindent

La STL es la librer�a est�ndar de plantillas. Sus principales componentes son contenedores, algoritmos, iteradores y objetos
funciones. Esta librer�a forma parte del Standard ISO C++. En la librer�a que implementa el seguimiento de puntos se ha
utilizado los contenedores t�picos (listas y vectores) y los iteradores que acceden a los datos (similar a los punteros).

Las principales estructura de almacenamiento se ha utilizado ya que aporta muchas ventajas y pocos inconvenientes:

\paragraph {Ventajas:}
\begin{enumerate}
 \item Facilidad de uso y flexibilidad.
  \noindent

    Las operaciones de inserci�n, extracci�n y modificaci�n son intuitivas y con una interfaz de programaci�n muy sencilla.
    Permite una gran variedad de operaciones y b�squedas de alto nivel dentro de la estructura.

 \item Optimizaci�n de recursos.
  \noindent

    Aleja al programador de toda la complejidad del manejo y conocimiento de la estructura subyacente. Por ello, se puede
    optimizar las operaciones sobre la estructura de manera transparente al programador.

 \item Fiabilidad.
  \noindent

    Se trata de una estructura ampliamente probada y que permite al programador centrarse en otros problemas que no sean el
    almacenamiento de datos, por lo que los esfuerzos de desarrollo y depuraci�n ir�n destinados en otro sentido.

\end{enumerate}

\paragraph {Inconvenientes:}
\begin{enumerate}
 \item Necesario un compilador de C++.
  \noindent

    Las STL son unas librer�as desarrolladas en C++, por lo que cualquier programa que las utilice tambi�n debe estar escrito
    en C++ y necesita un compilador de C++.

\end{enumerate}



% *) FUENTES DE INFORMACI�N
% Aqu� va la Bibliograf�a utilizada por el proyecto.
\begin{thebibliography}{15}

% Calibraci�n y contenido del detector de Harris
\bibitem{FAUG93}
Faugeras, Olivier, \emph{Three-Dimensional Computer Vision} The MIT Press.
(1993).

\bibitem{FAUG01}
Faugeras, Olivier and Quang-Tuan Luong and Theo Papadopoulo,
\emph{The geometric of multiples images: the law that govern the formation of multiple
images of a scene and some of their applications} The MIT Press.
(2001).

\bibitem{KANA95}
Kenichi Kanatani, \emph{Geometric Computation for Machine Vision} Oxford University Press.
(1995).

\bibitem{HART01}
Richard Hartley and Andrew Zisserman, \emph{Multiple View Geometry in Computer Vision} The MIT Press.
(2001).

\bibitem{ALVA01}
Alvarez, Luis and Cuenca, Carmelo and Mazorra, Luis, \emph{Signal Processing, Pattern Recognition
and Applications. IASTED.} Morphological Corner Detector. Application to Camera Calibration.
(2001).


% Ingenieria del Software
\bibitem{PRE01}
Roger S. Pressman, \emph{Ingenier�a del Software, quinta edici�n} McGraw-Hill Companies.
(2001).

\bibitem{GAM94}
Erich Gamma, Richard Helm, Ralph Johnson and John Vlissides, \emph{Design Patterns} Addison Wesley Publishing Company.
(1994).


% Programaci�n
\bibitem{HEI98}
Gregory L. Heileman, \emph{Estructuras de datos, algoritmos y programaci�n orientada a objetos.}
McGraw-Hill/Interamericana de Espa�a S.A.U. 1998.

\bibitem{MEY97}
Scott Meyers, \emph{Effective C++} Addison Wesley Publishing Company.
(1997).

\bibitem{WER94:1}
Josie Wernecke, \emph{The Inventor Mentor} Addison Wesley Publishing Company.
(1994).

\bibitem{WER94:2}
Josie Wernecke, \emph{The Inventor Toolmaker} Addison Wesley Publishing Company.
(1994).

\bibitem{WRI97}
Richard S. Wright Jr., \emph{Programaci�n en OpenGL} Anaya Multimedia.
(1997).


% UML
\bibitem{BOO00}
Grady Booch, James Rumbaugh and Ivar Jacobson, \emph{El lenguaje unificado de modelado} Addison Wesley Publishing Company.
(2000).

\bibitem{FOW97}
Martin Fowler and Kendall Scott, \emph{UML gota a gota} Addison Wesley Publishing Company.
(1997).


% Documentaci�n
\bibitem{DIL93}
Antoni Diller, \emph{Latex line by line} Wiley Professional Computing.
(1993).


\end{thebibliography}


% Termina el documento
\end{document}

