\chapter[Seguimiento de puntos]{Seguimiento de puntos singulares a trav�s de una secuencia de v�deo}

En este cap�tulo vamos a estudiar el funcionamiento del proceso de seguimiento de puntos singulares a
trav�s de una secuencia de v�deo. El seguimiento consiste en hallar la trayectoria de los puntos singulares 
a trav�s de los distintos frames. La trayectoria de un punto 3d consiste en una lista con las proyecciones 
de dicho punto en los distintos frames donde es visible.

Este cap�tulo se divide en tres partes: definici�n de conceptos, criterios de sucesi�n y la descripci�n del
proceso de seguimiento. El apartado de definici�n de conceptos es imprescindible para introducir al lector
en la terminolog�a que se maneja. Los criterios de sucesi�n definir�n el comportamiento que tendr� el proceso
de seguimiento.



\section{Proceso del seguimiento}
\noindent

\subsection{Conceptos}
\noindent

A lo largo de los siguientes apartados manejaremos una serie de conceptos por lo que necesitaremos
entenderlos perfectamente. A continuaci�n se definen dichos conceptos:


\begin{itemize}
\item Detector de esquinas:
\noindent
Herramienta software que nos identifica un conjunto de esquinas (extremos de curvatura) en una imagen.

\item Frame:
\noindent
Instant�nea que recoge informaci�n captada por una c�mara (ya sea de v�deo o de fotograf�a).
%Instant�nea extraida de un v�deo. En nuestro caso, ser� una determinada imagen de la secuencia de entrada.

\item Par�metros de Harris:
\noindent
El algoritmo de Harris es el detector de esquinas que utilizaremos el seguimiento de los puntos.
Sus par�metros han sido descritos en el cap�tulo anterior.

\item Punto sucesor:
\noindent
Punto continuaci�n de otro situado en el frame anterior que corresponde al mismo punto 3d.

\item Secuencia de puntos:
\noindent
Valor devuelto en el seguimiento, en que se identifica como una lista de puntos aquellas proyecciones de
frames consecutivos que correspondan al mismo punto 3d.

\item Secuencia de v�deo:
\noindent
Conjunto de frames desglosados uno independiente del otro.

\end{itemize}


