\chapter[Detalles de implementaci�n]{Detalles t�cnicos sobre la implementaci�n del proyecto}

En este cap�tulo se expone todo lo relacionado con los diferentes detalles t�cnicos de la implementaci�n del proyecto. Se
comienza introduciendo las decisiones m�s importantes que se han tomado a lo largo de las fases de dise�o, desarrollo y
pruebas.

Dada la naturaleza gr�fica del proyecto, se escogi� para la implementaci�n el sistema operativo Linux y su entorno {\it X
Window \/}, frente por ejemplo al entorno MS-Windows, tanto por la disponibilidad de m�quinas sobre las que trabajar, de las
cuales he obtenido gran experiencia, como de las ventajas en cuanto a fiabilidad y posibilidad de portar el c�digo a otras
implementaciones de S.O. que a su vez ofrecen, en su conjunto, un amplio abanico de plataformas hardware.

El XMegaWave es un entorno, con una interfaz gr�fica, para el procesamiento gr�fico de im�genes. El usuario puede interactuar
con la interfaz mediante rat�n y pop-up men�s. Adem�s se puede incluir en el entorno funciones creadas por propio usuario.
El XMegaWave puede trabajar con v�deos (en un formato especial ) y con im�genes, tanto en color como en escala de grises.
Est� orientado a estaciones UNIX con X11 y librer�as gr�ficas Motif.




\section{Uso de las STL (Standard Template Library)}
\noindent

La STL es la librer�a est�ndar de plantillas. Sus principales componentes son contenedores, algoritmos, iteradores y objetos
funciones. Esta librer�a forma parte del Standard ISO C++. En la librer�a que implementa el seguimiento de puntos se ha
utilizado los contenedores t�picos (listas y vectores) y los iteradores que acceden a los datos (similar a los punteros).

Las principales estructura de almacenamiento se ha utilizado ya que aporta muchas ventajas y pocos inconvenientes:

\paragraph {Ventajas:}
\begin{enumerate}
 \item Facilidad de uso y flexibilidad.
  \noindent

    Las operaciones de inserci�n, extracci�n y modificaci�n son intuitivas y con una interfaz de programaci�n muy sencilla.
    Permite una gran variedad de operaciones y b�squedas de alto nivel dentro de la estructura.

 \item Optimizaci�n de recursos.
  \noindent

    Aleja al programador de toda la complejidad del manejo y conocimiento de la estructura subyacente. Por ello, se puede
    optimizar las operaciones sobre la estructura de manera transparente al programador.

 \item Fiabilidad.
  \noindent

    Se trata de una estructura ampliamente probada y que permite al programador centrarse en otros problemas que no sean el
    almacenamiento de datos, por lo que los esfuerzos de desarrollo y depuraci�n ir�n destinados en otro sentido.

\end{enumerate}

\paragraph {Inconvenientes:}
\begin{enumerate}
 \item Necesario un compilador de C++.
  \noindent

    Las STL son unas librer�as desarrolladas en C++, por lo que cualquier programa que las utilice tambi�n debe estar escrito
    en C++ y necesita un compilador de C++.

\end{enumerate}

