\chapter{Resultados}

En cada uno de los cap�tulos anteriores se han comentado de manera exhaustiva los resultados emp�ricos obtenidos en cada etapa. En este �ltimo apartado vamos a globalizar el resultado final aplicado a una secuencia de v�deo.

% Descripci�n de las condiciones en la que se tom� la secuencia
Partimos de una secuencia de 120 im�genes reales de un despacho. Cada una de estas im�genes fueron tomadas por una c�mara digital con un zoom constante. La c�mara est� situada en el fondo del despacho y se va a ir desplazando paralelo a la pared. El movimiento seguido por la c�mara se asemeja a una recta. La diferencia entre dos im�genes consecutivas es peque�a. Al no tomarse esta secuencia con una c�mara de v�deo digital, se observar�n peque�o saltos entre im�genes por lo que el movimiento no ser� fluido.

La c�mara se ha situado encima de un soporte m�vil y es �ste quien se mueve. De esta manera se evita el cabeceo de la c�mara. La fuente de luz en la escena est� situada detr�s de la c�mara y la imagen tomadas son de buena calidad y de gran claridad. 

En las figuras \ref{fig_despacho_inicial_0_30}, \ref{fig_despacho_inicial_50_90} y \ref{fig_despacho_inicial_100_119} podemos observar una mini secuencia de seis im�genes. Estas im�genes son un subconjunto de secuencia de v�deo inicial (120). Cada una de estas im�genes pertenece a distintas partes de esta secuencia. En concreto se han tomado las im�genes de las posiciones 0, 30 50, 90, 100 y 119.


% % Secuencia Inicial
% \begin{figure}
%    \begin{center}
%      \includegraphics[width=14cm ]
% 	  {imagen_ini_1_30.ps}
%      \caption{De arriba a abajo tenemos las im�genes de las posiciones 0 y 30 de la secuencia, respectivamente.}
%      \label{fig_despacho_inicial_0_30}
%    \end{center}
% \end{figure}
% 

